\documentclass{article}
\usepackage[utf8]{inputenc}
\usepackage{graphicx}
\usepackage{caption}
\usepackage{hyperref}
\usepackage{tcolorbox}
\usepackage{xcolor}
\tcbuselibrary{theorems}

\newenvironment{myminipage}[1]{\minipage{#1} 
\captionsetup{width=\textwidth, name=Fig., labelfont={it,bf}}
}{\endminipage} 

\newtcbtheorem[no counter]{need}{Need}%
{ 
    sharp corners,
    colback=green!5,
    colframe=green!25,
    fonttitle=\bfseries,
    coltitle=black,
}{th}

\begin{document}

\begin{center}
\includegraphics[width=\textwidth]{Data/logocircle.png}
\title{Relazione Greenlivery}
\author{Flavio Gezzi, Matteo Zacchino}
\end{center}
\renewcommand{\contentsname}{Indice}

\maketitle
\tableofcontents
\newpage
\section{Introduzione}\par
Il progetto assegnato dal docente riguarda la realizzazione di un servizio legato al cibo e all'alimentazione. Le aree di interesse principali includono: \par
\begin{itemize}
    \item \textbf{Dieta}: gestione della dieta per uno stile di vita sano, ad esempio per motivi medici o personali, dieta per sportivi, gestione di ricettari, diario alimentare.\par
    \item \textbf{Green}: promozione di cambiamenti nelle abitudini alimentari per ridurre lo spreco di cibo e le risorse impiegate nella produzione, gestione intelligente di colture e allevamenti.\par
    \item \textbf{Cultura}: valorizzazione delle tradizioni alimentari, dei prodotti tipici locali, percorsi enogastronomici, slow food.\par
    \item \textbf{Aggregazione/Commercio}: servizi di ristorazione, bar, consegna della spesa o di pasti pronti.\par
    \item \textbf{Tecnologia}:  tele-dining (ad esempio, aperitivi a distanza), mukbang (guardare altre persone mangiare su YouTube), stampa di cibo, ecc. \par
\end{itemize}
\vspace{1cm}



    \addcontentsline{toc}{subsection}{\numberline{1.1}Argomento del progetto} \par
    \numberline{\fontsize{4mm}{1mm}\selectfont \textbf{1.1 Argomento del progetto}}\vspace{0.5cm}
    L'obiettivo del nostro progetto, denominato "Greenlivery", è sviluppare un'app di consegna che si concentra sulla sostenibilità e combatte lo spreco alimentare. La nostra app offre la possibilità di ordinare cibo da ristoranti locali con consegne effettuate in modo ecologicamente compatibile, contribuendo così a ridurre l'impatto ambientale delle consegne. Inoltre, siamo sensibili allo spreco alimentare e miriamo a soddisfare sia i clienti che i ristoratori. I ristoranti possono utilizzare la nostra app per ridurre lo spreco alimentare, offrendo ai clienti la possibilità di acquistare una "Food Magic Box", che consente loro di acquistare porzioni di cibo in eccesso a prezzi convenienti. In questo modo, i ristoranti possono ridurre i costi e aumentare i profitti. 
    \vspace{1cm}
    
    \addcontentsline{toc}{subsection}{\numberline{1.2}Analisi app competitor} \par
    \newpage
    \numberline{\fontsize{4mm}{1mm}\selectfont\textbf{1.2 Confronto con app simili}}\vspace{0.5cm}
    Analizzando le principali app di food delivery, è emerso che tutte posseggono sia un sito web che un'app per smartphone, ci permettono di geolocalizzare il nostro indirizzo e ci consigliano i ristoranti. Inoltre la user-experience è ottima su tutte quelle analizzate, offrono tutte un buon supporto clienti ed hanno tutte un'identità di brand definita. Per questo la nostra app dovrà avere tutte queste caratteristiche oltre alle novità che implementeremo. Tuttavia, analizzando singolarmente le peculiarità delle app, abbiamo notato che ognuna di esse ha dei punti di forza e dei punti deboli che le differenzia tra loro.\vspace{0,7cm}
\begin{itemize}
        \item \textbf{App n1 - \href{https://www.justeat.it}{JustEat}:} Ha un’interfaccia intuitiva con le categorie e la ricerca in primo piano, non permette però di visualizzare il nome, la foto del rider. Non consente, inoltre, di chattare con esso o con il ristorante.\vspace{0.3cm}

        \item \textbf{App n2 - \href{https://glovoapp.com/it/it/}{Glovo}:} Ha un'interfaccia divisa in macrocategorie che ci offre servizi concorrenziali, come inviare pacchi da un indirizzo all’altro o ordinare farmaci. É possibile visualizzare delle informazione sul rider come la posizione in tempo reale e parlarci tramite live chat. Offre anche la possibilità di accumulare credito sull’app invitando degli amici ad iscriversi. É possibile inoltre recensire il ristorante, l’ordine ricevuto e il rider che lo ha consegnato.\vspace{0.3cm}

        \item \textbf{App n3 - \href{https://deliveroo.it/it/}{Deliveroo}:} Come JustEat ci offre in homepage le varie categorie, le promozioni in corso e la ricerca dei ristoranti. Ci permette di visualizzare informazioni e posizione del rider ed ha un servizio di recensione come quello di Glovo. La peculiarità di questa app è poter aggiungere alcuni ristoranti tra i preferiti e visualizzarli facilmente tramite un pulsante in homepage. Aprendo le categorie è possibile filtrare i ristoranti in base a regimi alimentari (Vegano, senza glutine, kosher ...).\vspace{0.3cm}

        \item \textbf{App n4 - \href{https://www.ubereats.com/it}{UberEats}:} Interfaccia intuitiva, ma con meno categorie rispetto ai concorrenti. Permette di visualizzare tutte le informazioni e la posizione del fattorino in tempo reale, consentendo di valutare non solo il ristorante e il fattorino, ma anche ogni singolo prodotto dell'ordine.\vspace{0.3cm}

    

\end{itemize}
\newpage
Abbiamo condotto un'analisi approfondita sulle quattro applicazioni che sono emerse come le più utilizzate tra gli utenti che hanno risposto al nostro questionario. Di seguito sono riportati i grafici che mostrano i risultati ottenuti.
\begin{center}
    \includegraphics[width=\textwidth]{Data/Grafici/App_competitor.png} 
    \includegraphics[width=\textwidth]{Data/Grafici/App_competitor_utilizzate.png}
\end{center}



\newpage
\section{Needfinding} 
In questa fase di sviluppo l’obiettivo è comprendere le necessità degli utenti per capire le funzionalità da implementare nell’applicazione. Per farlo abbiamo utilizzato due metodi: il questionario e le interviste.
    \vspace{1cm}\addcontentsline{toc}{subsection}{\numberline{2.1}Interviste} \par
    \numberline{\fontsize{4mm}{1mm}\selectfont \textbf{2.1 Interviste}}\vspace{0.5cm}
        \par Abbiamo creato due versioni delle interviste. Dopo aver somministrato la prima versione agli intervistati, abbiamo individuato i problemi e abbiamo elaborato la seconda versione con domande più chiare e comprensibili. \par Le interviste sono state poste per lo più a colleghi di università, amici e colleghi di lavoro per una durata media di 7 minuti.\par
        \par Il file con le interviste è disponibile \textbf{\href{https://wind-blob-6b0.notion.site/Interviste-de02a85b8b634f7687534d2064010b4a}{a questo sito}.}\par
    \vspace{1cm}\addcontentsline{toc}{subsubsection}{\numberline{2.1.1}Domande}\par
    \numberline{\fontsize{4mm}{1mm}\selectfont \textbf{2.1.1 Domande}}\vspace{0.5cm}
    \begin{enumerate}
    
    \item \textit{Ordini mai a domicilio?}
        \begin{enumerate}
            \item Se no, perché?
        \end{enumerate}
        \begin{tcolorbox}[
            colframe=green,  % Colore del contorno del riquadro
            colback=white,  % Colore di sfondo del riquadro
            sharp corners,  % Angoli affilati
            boxrule=0.5pt,  % Spessore del contorno
            left=5pt,  % Distanza sinistra del testo dal riquadro
            right=5pt,  % Distanza destra del testo dal riquadro
            top=5pt,  % Distanza superiore del testo dal riquadro
            bottom=5pt  % Distanza inferiore del testo dal riquadro
        ]
        Attraverso questa domanda, possiamo valutare se l'intervistato potrebbe rappresentare un potenziale utente della nostra applicazione. Siamo consapevoli del fatto che molte delle principali applicazioni concorrenti non sono in grado di fornire copertura in determinate aree o città. Nel caso in cui l'intervistato non abbia mai effettuato ordini di cibo a domicilio, vorremmo gentilmente chiedergli le ragioni di questa scelta.
        \end{tcolorbox}
    \newpage
    \item \textit{Conoscete e avete mai utilizzato un'app di food delivery?}
        \begin{enumerate}
            \item Quali?
            \item Quanto spesso le usi?
        \end{enumerate}
        \begin{tcolorbox}[
            colframe=green,  % Colore del contorno del riquadro
            colback=white,  % Colore di sfondo del riquadro
            sharp corners,  % Angoli affilati
            boxrule=0.5pt,  % Spessore del contorno
            left=5pt,  % Distanza sinistra del testo dal riquadro
            right=5pt,  % Distanza destra del testo dal riquadro
            top=5pt,  % Distanza superiore del testo dal riquadro
            bottom=5pt  % Distanza inferiore del testo dal riquadro
        ]
        Abbiamo adottato l'approccio di sondare gli intervistati per valutare la loro familiarità e l'utilizzo delle applicazioni di food delivery. Questa strategia ci consente di ottenere una comprensione più accurata del livello di utilizzo e conoscenza di tali servizi, nonché di gestire in modo più efficiente le domande durante le interviste, evitando di porre quesiti a cui gli intervistati non sarebbero in grado di rispondere. Nel caso di risposte affermative, abbiamo inoltre approfondito la frequenza con cui effettuano ordini e le preferenze in termini di applicazioni specifiche.
        \end{tcolorbox}
    \item \textit{Quali funzionalità preferisci dell'app che utilizzi?}
        \begin{tcolorbox}[
            colframe=green,  % Colore del contorno del riquadro
            colback=white,  % Colore di sfondo del riquadro
            sharp corners,  % Angoli affilati
            boxrule=0.5pt,  % Spessore del contorno
            left=5pt,  % Distanza sinistra del testo dal riquadro
            right=5pt,  % Distanza destra del testo dal riquadro
            top=5pt,  % Distanza superiore del testo dal riquadro
            bottom=5pt  % Distanza inferiore del testo dal riquadro
        ]
        Questa domanda è rivolta agli utenti che hanno già utilizzato almeno una volta le applicazioni di food delivery. La finalità è analizzare le funzionalità che sono state particolarmente apprezzate dagli utenti, al fine di poterle integrare nella nostra stessa applicazione.
        \end{tcolorbox}
    \item \textit{Quali potrebbero essere delle funzionalità che ti potrebbero tornare utilizzi ma che non sono implementate?}
        \begin{tcolorbox}[
            colframe=green,  % Colore del contorno del riquadro
            colback=white,  % Colore di sfondo del riquadro
            sharp corners,  % Angoli affilati
            boxrule=0.5pt,  % Spessore del contorno
            left=5pt,  % Distanza sinistra del testo dal riquadro
            right=5pt,  % Distanza destra del testo dal riquadro
            top=5pt,  % Distanza superiore del testo dal riquadro
            bottom=5pt  % Distanza inferiore del testo dal riquadro
        ]
        La presente domanda mira a raggiungere gli stessi obiettivi della precedente, ma si concentra sulla comprensione degli interessi degli utenti e sulla scoperta di funzionalità che non sono attualmente disponibili nelle applicazioni concorrenti.
        \end{tcolorbox}
    \item \textit{Ti farebbe comodo conoscere i valori nutrizionali(calorie, proteine, carboidrati, grassi) delle pietanze che ordini?}
        \begin{enumerate}
            \item La troveresti utile se dovessi seguire una dieta?
        \end{enumerate}
        \begin{tcolorbox}[
            colframe=green,  % Colore del contorno del riquadro
            colback=white,  % Colore di sfondo del riquadro
            sharp corners,  % Angoli affilati
            boxrule=0.5pt,  % Spessore del contorno
            left=5pt,  % Distanza sinistra del testo dal riquadro
            right=5pt,  % Distanza destra del testo dal riquadro
            top=5pt,  % Distanza superiore del testo dal riquadro
            bottom=5pt  % Distanza inferiore del testo dal riquadro
        ]
        Attraverso questa domanda, abbiamo cercato di valutare l'interesse degli utenti per i principali valori nutrizionali di ogni piatto, anche considerando coloro che seguono particolari regimi alimentari o diete.
        \end{tcolorbox}
    \newpage
    \item \textit{Vista l'attuale situazione climatica globale, favoriresti un servizio di consegna a domicilio completamente Green? Se si, perché? }
        \begin{enumerate}
            \item Anche se questo dovesse comportare dei costi leggermente maggiori?
        \end{enumerate}
        \begin{tcolorbox}[
            colframe=green,  % Colore del contorno del riquadro
            colback=white,  % Colore di sfondo del riquadro
            sharp corners,  % Angoli affilati
            boxrule=0.5pt,  % Spessore del contorno
            left=5pt,  % Distanza sinistra del testo dal riquadro
            right=5pt,  % Distanza destra del testo dal riquadro
            top=5pt,  % Distanza superiore del testo dal riquadro
            bottom=5pt  % Distanza inferiore del testo dal riquadro
        ]
        Attraverso questa domanda, abbiamo cercato di valutare l'interesse degli utenti per i principali valori nutrizionali di ogni piatto, anche considerando coloro che seguono particolari regimi alimentari o diete.
        \end{tcolorbox}
\end{enumerate}
\vspace{1cm}\addcontentsline{toc}{subsubsection}{\numberline{2.1.2}Resoconto interviste}\par
\numberline{\fontsize{4mm}{1mm}\selectfont \textbf{2.1.2 Resoconto interviste}}\vspace{0.5cm}
    \par Le interviste hanno dimostrato di essere estremamente utili, poiché ci hanno consentito di comprendere l'ampia diffusione delle applicazioni di food delivery, specialmente tra i giovani che sono stati il principale gruppo intervistato. Queste interviste ci hanno permesso di identificare le funzionalità più apprezzate dagli utenti, che abbiamo successivamente integrato nella nostra applicazione. Inoltre, ci hanno fornito preziose informazioni riguardo a funzionalità ancora assenti in altre app ma che risultano di interesse per gli intervistati. Attraverso le interviste, abbiamo riscontrato che il tema dell'ecosostenibilità, rappresentato ad esempio dalle consegne "green", è ben accolto dagli intervistati. Inoltre, abbiamo notato che l'inclusione dei valori nutrizionali per ciascun pasto nell'applicazione è un'opzione gradita dalla maggioranza degli intervistati, a condizione che non comporti costi aggiuntivi.\par


\newpage
\vspace{2cm}\addcontentsline{toc}{subsection}{\numberline{2.2}Questionari} \par 
\numberline{\fontsize{4mm}{1mm}\selectfont \textbf{2.2 Questionari}}\vspace{0.5cm}
\par Abbiamo condotto un questionario che permette di raggiungere un numero di partecipanti decisamente maggiore rispetto alle interviste, per confermare se i need emersi sono verificati.\par Il questionario con Google Form consente anche di visualizzare in modo più preciso le preferenze degli utenti tramite grafici significativi.
\par La versione 1 del questionario è disponibile \textbf{\href{https://forms.gle/pBWCBfAxsjULCZRj6}{a questo sito}.}
\par La versione 2 del questionario è disponibile \textbf{\href{https://docs.google.com/forms/d/1ToAwRVi9a8q0_68hbEI7O8ORtodt_uTHuwU9ZtPfk1Q/edit}{a questo sito}.}
\par \vspace{1cm}
Nella seconda versione abbiamo ottenuto \textbf{158} risposte che andremo ad analizzare con l'aiuto dei grafici visualizzati su Google Moduli.\par \vspace{1cm}
\par Dopo aver constatato che solo l'11\% non ha mai effettuato un ordine, abbiamo indagato sul motivo per comprendere se fosse possibile fornire un servizio che li spingesse a ordinare cibo a domicilio.
\vspace{0.5cm}
\par 
\includegraphics[width=\textwidth]{Data/Grafici/Perche_non_ordina.png}
\includegraphics[width=\textwidth]{Data/Grafici/ordinare_cibo.png}
\par Abbiamo anche esaminato le risposte sulle percentuali dei mezzi di consegna per comprendere quanti utenti utilizzano le app per il food delivery.\par \vspace{1cm}
\includegraphics[width=\textwidth]{Data/Grafici/Funzionalità_fondamentali.png}\par \newpage
In base alle risposte raccolte, identifichiamo quali sono le caratteristiche che catturano immediatamente l'attenzione degli utenti.
\par \begin{itemize}
    \item Tracciamento dell'ordine;
    \item Abbinamenti consigliati dal ristorante;
    \item Chattare con il rider;
    \item Visualizzare gli sconti disponibili;
\end{itemize}   \vspace{1cm} \par
\includegraphics[width=\textwidth]{Data/Grafici/influenza_recensioni.png}
\par Abbiamo posto questa domanda dopo una minuziosa analisi dei concorrenti. Volevamo valutare l'impatto delle recensioni sulla scelta di un ristorante e determinare se fosse opportuno implementare un sistema di recensioni semplice o approfondito come quello di UberEats.
    \par \vspace{1cm}\includegraphics[width=\textwidth]{Data/Grafici/val_nutrizionali.png}\par
L'idea del controllo dei valori nutrizionali è emersa dalle interviste, essendo una funzionalità innovativa abbiamo inserito questa domanda nel questionario. È evidente che la maggioranza gradirebbe questa funzione, per questo è stata implementata nel prototipo.\par \vspace{1cm}
\includegraphics[width=\textwidth]{Data/Grafici/situazione_climatica.png}\par
Nelle interviste questa è stata la domanda più controversa e con più equilibrio tra i favorevoli e i contrari. Con il questionario ci siamo resi conto che questa idea non poteva essere implementata poichè solo il 25\% sarebbe disposto a pagare un sovrapprezzo. \par \vspace{1cm}
\includegraphics[width=\textwidth]{Data/Grafici/too_good_too_go.png}\par
.\par \vspace{1cm}
\includegraphics[width=\textwidth]{Data/Grafici/totg_delivery.png} \par
Abbiamo scoperto che uno degli intervistati aveva l'idea di implentare le food magic box, il surplus invenduto del ristorante a prezzo scontato. I risultati mostrano che questa soluzione per ridurre lo spreco alimentare non è diffusa. Dopo aver esaminato la concorrenza, "Too Good To Go", abbiamo scoperto che non offre la consegna a domicilio. I risultati dei questionari hanno mostrato che se fosse disponibile questo servizio, la maggioranza lo utilizzerebbe.
\vspace{1.5cm}
\begin{need}{}{theoexample}
    Dopo aver analizzato i risultati ottenuti dal questionario abbiamo riscontrato i seguenti need:
    \begin{enumerate}
        \item Risoluzione eventuali disguidi alla consegna.
        \item Combattere lo spreco alimentare.
        \item Visualizzazione in tempo reale della posizione del Rider.
        \item Avere la possibilità di recensire sia il ristorante, sia la pietanza singola.
        \item Avere a disposizione tutti i principali metodi di pagamento.
        \item Avere a disposizione un filtro per i \textit{valori nutrizionli} .
    \end{enumerate}
    \end{need}

    \vspace{1cm}
\vspace{4cm}
\pagebreak
\section{Storyboard} \par
Le seguenti storyboard illustrano le principali task che sono state individuate:


\begin{figure}[htpb]

\begin{minipage}{0.40\textwidth}
    \includegraphics[width=\textwidth]{Data/StoryBoard/Effetuare_ordine.png}
    \caption[Prima figura]{Effettuare un ordine} \label{fig:1}
\end{minipage}
\hspace{2cm}
\begin{minipage}{0.40\textwidth}
    \includegraphics[width = \textwidth]{Data/StoryBoard/risoluzione disguidi alla cosegna.png}
    \caption[Seconda figura]{Usufruire dei servizi per avere una consegna precisa}\label{fig:2}
\end{minipage}
\begin{minipage}{0.40\textwidth}
    \includegraphics[width=\textwidth]{Data/StoryBoard/Recensione_ristorante.png}
    \caption[Terza figura]{Fornire valutazione ristorante e pietanza}\label{fig:3}
\end{minipage}
\hspace{2cm}
\begin{minipage}{0.40\textwidth}
    \includegraphics[width=\textwidth]{Data/StoryBoard/Filtro_calorie.jpeg}
    \caption[Quarta figura]{Usufruire del segna-calorie se si segue una dieta}\label{fig:4}
\end{minipage}
\end{figure}

\vspace{4cm}
\section{Prototipi}
    Ciascun task è stato testato da un gruppo di 4-5 persone e l'approccio al prototyping è stato di tipo evolutivo. Ad ogni iterazione,vengono proposte modifiche al fine di migliorare il prototipo e risolvere i problemi evidenziati nella versione precedente.\par
    \vspace{1cm}\addcontentsline{toc}{subsection}{\numberline{4.1}Paper Prototyping}
    \numberline{\fontsize{4mm}{1mm}\selectfont \textbf{4.1 Paper Prototyping}}\vspace{0.5cm}
    \par Abbiamo creato tramite Figma-Jam una bozza del prototipo disegnando le schermate principali che successivamente abbiamo implementato.
    \par Per visualizzare il paper prototyping \textbf{\href{https://t.ly/AbuU}{clicca qui}.}

    \vspace{1cm}\addcontentsline{toc}{subsection}{\numberline{4.2}Prototipo Versione 1}
    \numberline{\fontsize{4mm}{1mm}\selectfont \textbf{4.2 Prototipo Versione 1}}\vspace{0.5cm}
    \par La prima versione del prototipo presentava delle criticità di usabilità che sono state risolte nella versione 2.
    \par Per visualizzare la versione 1 del prototipo \textbf{\href{https://t.ly/g4ZQ}{clicca qui}.}

    \vspace{1cm}\addcontentsline{toc}{subsection}{\numberline{4.3}Prototipo Versione 2}
    \numberline{\fontsize{4mm}{1mm}\selectfont \textbf{4.3 Prototipo Versione 2}}\vspace{0.5cm}
    \par Dopo aver esaminato gli User Test della prima versione abbiamo trovato e corretto gli errori di collegamento tra le schermate. 
    \par Per visualizzare la versione finale del prototipo \textbf{\href{https://t.ly/ecm4}{clicca qui}.}
    
\section{UserTest} \par\vspace{0.5cm}

\par Dopo aver terminato ciascuna versione del prototyping abbiamo effettuato un test di usabilità con diverse persone.
Il link per visualizzare tutti gli User Test è disponibile \textbf{\href{https://www.notion.so/User-Testing-03a9697158cb4f93b3439fe694f33ac9}{qui }.}
\section{Expert Evaluation}
\par Abbiamo successivamente effettuato un Expert Evaluation per testare l'usabilità del prototipo.
Il link per visualizzarlo è disponibile \textbf{\href{https://www.notion.so/Expert-Evaluation-7bc49cae89414742a51b71f6faebecdd}{qui }.}

\vspace{1cm}

\newpage
\section{Fattibilità} \par\vspace{0.5cm}
\par Abbiamo analizzato la fattibilità del progetto e abbiamo scoperto che è possibile realizzarlo. Abbiamo individuato i seguenti punti: \par
\begin{enumerate}
    \item GPS - Utile per localizzare l’utente e fornire i ristoranti nelle vicinanze e per visualizzare lo stato di consegna in tempo reale. Questo è possibile tramite il servizio di geolocalizzazione integrato nello smartphone.
    \item Database - Per immagazzinare i dati di login degli utenti, i ristoranti e tutti i prodotti. È facilmente implementabile e gestibile tramite un comune DBMS.
    \item Server - Per gestire le connessioni tra i client e l’applicazione, per gestire la messaggistica istantanea tra utente e ristorante/help center/rider.
    \item Algoritmo - Per mostrare nella homepage i ristoranti più vicini all’utente in maniera ordinata e calcolare i prodotti più venduti di ogni ristorante partner ed inserirli nella categoria popolari.
    \item Sviluppare l’applicazione in linguaggio Swift essendo stata ideata per dispositivi IOS.
    \item Utilizzare un servizio di call center per fornire assistenza ai clienti.
\end{enumerate}


\end{document}
